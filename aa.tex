\documentclass[10pt,a4paper,oneside]{book}
\usepackage[utf8]{inputenc}
\usepackage[french]{babel}
\usepackage[T1]{fontenc}
\usepackage{amsmath,mathrsfs,amsfonts,amssymb}
\usepackage{tikz,tkz-tab,pgfplots}
\pgfplotsset{compat=1.11}
\usepackage{qrcode}
\usepackage{pythontex}
\usepackage[left=1cm,right=1cm,top=1.5cm]{geometry}
\usepackage{hyperref}


\usepackage{titlesec}
\usepackage{ntheorem}
\usepackage{tcolorbox}
\usepackage{multicol}
\usepackage{pst-plot,pst-tree,pst-node}

\theoremstyle{break}
\newtheorem{exercice}{Exercice}[section]
\definecolor{monbleu}{RGB}{50,71,110}
\definecolor{bleusol}{RGB}{176,196,222}
\makeatletter
\newcommand{\enableopenany}{%
\@openrightfalse%
}
\makeatother

\newcommand\titre[1]{\begin{flushright}\color {monbleu}{\textbf{\huge{#1}}}\end{flushright}\vspace*{-0.7cm}}
\newtcolorbox{boxsection}{colback=monbleu,coltext=white,colframe=white}
\newcommand\masection[1]{\begin{boxsection}\vspace*{-0.2cm}\Large{#1}\vspace*{-0.2cm}\end{boxsection}\vspace*{-0.2cm}}

\titleformat*{\section}{\masection}
\titleformat*{\subsection}{ \bfseries \large \color{monbleu}}

\newtcolorbox{boxexo}{colframe=monbleu}
\newcommand\exo[1]{\begin{boxexo}\begin{exercice}{#1}\end{exercice}\end{boxexo}}

\newtcolorbox{boxsol}{colback=bleusol,colframe=white}
\newcommand\sol[1]{\begin{boxsol}{#1} \end{boxsol}}


\title{Automatex}
\author{Arnaud Meistermann}

\begin{document}
\begin{pycode}
from sympy import *
x, y, z, t ,h,q,a,b,c,d = symbols('x y z t h q a b c d')

def saisie_fonction(txt):
    liste=['x','a','b','y','z','q','e','i','ln','n','\pi','t','sqrt','(']
    liste2=["a","b",'y','x','n','t','q','z']

    for i in range(0,10):
        for lettre in liste:
            txt=txt.replace(str(i)+lettre,str(i)+"*"+lettre)

    txt=txt.replace(")(",")*(")
    txt=txt.replace("ln","log")
    for lettre in liste2:
        if lettre!="t" or "sqrt" not in txt:
            txt=txt.replace(lettre+"(",lettre+"*(")
        txt=txt.replace(")"+lettre,")*"+lettre)
        txt=txt.replace("e^"+lettre,"exp("+lettre+")")
    txt=txt.replace("i","I")
    txt=txt.replace("e^","exp")
    txt=txt.replace("^","**")
    txt=txt.replace("{","(")
    txt=txt.replace("}",")")
    txt=txt.replace(" ","")
    txt=txt.replace(")sqrt",")*sqrt")
    txt=txt.replace(")e",")*e")
    txt=txt.replace(")log",")*log")
    txt=txt.replace("over","/")

    return (txt)


def formate(txt):
    txt=txt.replace("\\frac","\\dfrac")
    if not("left" in txt):
        txt=txt.replace("(","\\left(")
    if not "right" in txt:
        txt=txt.replace(")","\\right)")
    txt=txt.replace("\\equiv","\\Leftrightarrow")
    txt=txt.replace("\\infty","+\\infty")
    txt=txt.replace("log",'ln')
    txt=txt.replace("--","+")
    txt=txt.replace("+-","-")
    txt=txt.replace("-+","-")
    txt=txt.replace("++","+")
    txt=txt.replace("\\vec","\\overrightarrow")
    txt=txt.replace(">="," \\ge ")
    txt=txt.replace("<="," \\le ")
    return txt

def change_signe(txt):
    txt=txt.replace("+","plus")
    txt=txt.replace("-","+")
    txt=txt.replace("plus","-")
    return txt

def signe_infini(a):
    if a=="\\infty":
        return "+"+a
    else:
        return a


def signeplus(fonction):
    func=latex(fonction)
    if func[0]!="-":
        return "+"
    else:
        return ""

def ajouteplus(nbre):
    if nbre>=0:
        return "+"+latex(nbre)
    else:
        return latex(nbre)
        
def parenthese(nbre):
    if nbre<0:
        return "("+latex(nbre)+")"
    else:
        return latex(nbre)

def variable(fonction):
    var=Symbol("x")
    if ("n" in fonction) and ("ln" not in fonction):
        var=Symbol("n")
    if "t" in fonction and "sqrt" not in fonction:
        var=Symbol("t")
    if "q" in fonction and "sqrt" not in fonction:
        var=Symbol("q")
    if "p" in fonction and "exp" not in fonction:
        var=Symbol("p")
    return var

def str2list(txt):
    L1=txt.split()
    L2=[]
    for elt in L1:
        L2.append(float(elt))
    return L2


\end{pycode}

\begin{pycode}
x, y, z, t ,h,q,a,b,c= symbols('x y z t h q a b c')


def equation_degre1(gauche,droite,symbole):
    gauche=saisie_fonction(gauche)
    droite=saisie_fonction(droite)
    var=variable(gauche)
    g=sympify(gauche)
    d=sympify(droite)
    if symbole !="=":
        ineq=sympify(gauche+symbole+droite)

    if (degree(d-g,var)>1) or degree(d-g,var)==0:
        txt="Ce n'est pas une équation ou une inéquation du premier degré"
    else:
        txt="$"+latex(g)+symbole+latex(d)+"$\\\\"
        g1=sympify(gauche)
        d1=sympify(droite)
        if (latex(g)!=latex(g1) or latex(d)!=latex(d1)):
            txt+="$ \\Leftrightarrow "+latex(g1)+symbole+latex(d1)+"$\\\\"

        g2=g1-g1.replace(var,0)
        d2=d1
        if degree(d,var)==1:
            d2=d1-LT(d1)

        if degree(d,var)!=0 or g.replace(var,0)!=0:
            txt+="$ \\Leftrightarrow "+latex(g2)
            if degree(d,var)==1:
                txt+="-"+latex(LT(d1))
            txt+=symbole+latex(d2)
            if g1.replace(var,0)!=0:
                txt+="-"+latex(g1.replace(var,0))
            txt+="$\\\\"

            if degree(d,var)==1:
                g2=g2-LT(d1)
            d2=d2-g1.replace(var,0)
            txt+="$ \\Leftrightarrow "+latex(g2)+symbole+latex(d2)+"$\\\\"
        if degree(g2,var)!=1:
            if symbole=="=":
                if d2==0:
                    txt1="L'ensemble solution est : $S=\\mathbb{R}$"
                else:
                    txt1="L'ensemble solution est : $S=\\emptyset$"
            else:
                txt1="L'ensemble solution est : $S="+latex(solve_univariate_inequality(ineq,var,relational=False))+"$"
        else:
            if LC(g2)<0:
                if ">" in symbole:
                    symbole=symbole.replace(">","<")
                else:
                    symbole=symbole.replace("<",">")
            if LC(g-d)!=1:
                txt+="$ \\Leftrightarrow "+latex(g2/LC(g2))+symbole+"\\dfrac{"+latex(d2)+"}{"+latex(LC(g2))+"}$\\\\"
                if latex(d2/LC(g2))!="\\frac{"+latex(d2)+"}{"+latex(LC(g2))+"}":
                    txt+="$ \\Leftrightarrow "+latex(g2/LC(g2))+symbole+latex(d2/LC(g2))+"$\\\\"
            if symbole=="=":
                txt1="L'ensemble solution est S=$\\left\\{"+latex(d2/LC(g2))+"\\right\\}$"
            else:

                sol=latex(solve_univariate_inequality(ineq,var,relational=False))
                sol=sol.replace("(","]")
                sol=sol.replace(")","[")
                txt1="L'ensemble solution est S=$"+sol+"$"


    return(formate(txt),formate(txt1))


def tableau_signe_produit(f1,f2):
    f1,f2=sympify(saisie_fonction(f1)),sympify(saisie_fonction(f2))
    a1,b1=LC(f1),f1.replace(x,0)
    a2,b2=LC(f2),f2.replace(x,0)
    alpha1,alpha2=-b1/a1,-b2/a2
    ligne0="$-\\infty$, $"+latex(alpha1)+"$, $"+latex(alpha2)+"$, $+\\infty$"
    ligne1=" ,-,z,+,t,+, "
    ligne2=" ,-,t,-,z,+,"
    ligne3=" ,+,z,-,z,+, "
    if alpha1>alpha2:
        ligne0="$-\\infty$, $"+latex(alpha2)+"$, $"+latex(alpha1)+"$, $+\\infty$"
        ligne1,ligne2=ligne2,ligne1
    if a1<0:
        ligne1=change_signe(ligne1)
        ligne3=change_signe(ligne3)
    if a2<0:
        ligne2=change_signe(ligne2)
        ligne3=change_signe(ligne3)

    f1=latex(a1)+"x+"+latex(b1)
    f2=latex(a2)+"x+"+latex(b2)
    f1f2="("+f1+")("+f2+")"
    txt="\\resizebox{.7\\textwidth}{!}{"
    txt+="\\begin{tikzpicture}"
    txt+="\\tkzTabInit[lgt=5]{$x$ / 1 , $"+f1+"$ / 1,$"+f2+"$ / 1,$"+f1f2+"$ / 1 }{"+ligne0+"}"
    txt+="\\tkzTabLine{"+ligne1+"}"
    txt+="\\tkzTabLine{"+ligne2+"}"
    txt+="\\tkzTabLine{"+ligne3+"}"
    txt+="\\end{tikzpicture}}"
    return formate(txt)

def tableau_signe_quotient(f1,f2):
    f1,f2=sympify(saisie_fonction(f1)),sympify(saisie_fonction(f2))
    a1,b1=LC(f1),f1.replace(x,0)
    a2,b2=LC(f2),f2.replace(x,0)
    alpha1,alpha2=-b1/a1,-b2/a2
    ligne0="$-\\infty$, $"+latex(alpha1)+"$, $"+latex(alpha2)+"$, $+\\infty$"
    ligne1=" ,-,z,+,t,+, "
    ligne2=" ,-,t,-,z,+,"
    ligne3=" ,+,z,-,d,+,"
    if alpha1>alpha2:
        ligne0="$-\\infty$, $"+latex(alpha2)+"$, $"+latex(alpha1)+"$, $+\\infty$"
        ligne1,ligne2=ligne2,ligne1
        ligne3=" ,+,d,-,z,+, "

    if a1<0:
        ligne1=change_signe(ligne1)
        ligne3=change_signe(ligne3)
    if a2<0:
        ligne2=change_signe(ligne2)
        ligne3=change_signe(ligne3)
    f1=latex(a1)+"x+"+latex(b1)
    f2=latex(a2)+"x+"+latex(b2)
    f1f2="\\dfrac{"+f1+"}{"+f2+"}"
    txt="\\resizebox{.7\\textwidth}{!}{"
    txt+="\\begin{tikzpicture}"
    txt+="\\tkzTabInit[lgt=3]{$x$ / 1 , $"+f1+"$ / 1,$"+f2+"$ / 1,$"+f1f2+"$ / 1 }{"+ligne0+"}"
    txt+="\\tkzTabLine{"+ligne1+"}"
    txt+="\\tkzTabLine{"+ligne2+"}"
    txt+="\\tkzTabLine{"+ligne3+"}"
    txt+="\\end{tikzpicture}}"
    return formate(txt)

def ineq_produit(f1,f2,symbole):
    f1,f2=saisie_fonction(f1),saisie_fonction(f2)
    txt=equation_degre1(f1,"0",'>')[0].replace("\\\\","")+"\\\\"
    txt+=equation_degre1(f2,"0",'>')[0].replace("\\\\","")+"\\\\"
    txt+="On obtient ainsi le tableau suivant : \\\\"
    txt+=tableau_signe_produit(f1,f2)+"\\\\"
    f="("+latex(f1)+")*("+latex(f2)+")"
    txt+=sol_ineq(f+symbole+"0")[0]+sol_ineq(f+symbole+"0")[1]
    return txt

def ineq_quotient(f1,f2,symbole):
    f1,f2=saisie_fonction(f1),saisie_fonction(f2)
    txt=equation_degre1(f1,"0",'>')[0].replace("\\\\","")+"\\\\"
    txt+=equation_degre1(f2,"0",'>')[0].replace("\\\\","")+"\\\\"
    txt+="On obtient ainsi le tableau suivant : \\\\"
    txt+=tableau_signe_quotient(f1,f2)+"\\\\"
    f="("+latex(f1)+")/("+latex(f2)+")"
    txt+=sol_ineq(f+symbole+"0")[0]+sol_ineq(f+symbole+"0")[1]
    return txt

def sol_ineq(ineq):
    ineq=sympify(saisie_fonction(ineq))
    txt1="L'ensemble solution de $"+latex(ineq)+"$ est "
    txt="$"+latex(solve_univariate_inequality(ineq,x,relational=False))+"$"
    txt=txt.replace("(","]")
    txt=txt.replace(")","[")
    return formate(txt1),formate(txt)

def borne_intervalle(i):
    t=[]
    if i.is_EmptySet:
        return t
    else:
        if i.inf==-oo:
            t.append(-oo)
        for n in i.boundary:
            t.append(n)
        if i.sup==oo:
            t.append(+oo)
        return t

def liste_x_signe(f,a,b):
    var=variable(f)
    f=factor(sympify(f))
    bornes_plus=borne_intervalle(solve_univariate_inequality(f>0,var, relational=False).intersect(Interval(a,b)))
    bornes_moins=borne_intervalle(solve_univariate_inequality(f < 0, var, relational=False).intersect(Interval(a,b)))
    tab=list(set(bornes_plus+bornes_moins))
    return(sorted(tab))


def signe_f(f,a,b):
    var=variable(f)
    f=factor(sympify(f))
    I=solve_univariate_inequality(f>0,var, relational=False).intersect(Interval(a,b))
    if I.is_EmptySet:
        return "-"
    else:
        return "+"


def est_val_interdite(f,a):
    var=variable(f)
    f=sympify(f)
    bool=True
    if a==oo or a==-oo:
        image=limit(f,var,a)
    else:
        image=f.subs(var,a)
    try :
        if image!=zoo:
            bool=False
    except:
        if  image.is_real :
            bool= False
    return bool


def tabsigne(f,a,b):
        var=variable(f)
        f=saisie_fonction(f)
        f1=sympify(f)
        txt="\n \\begin{tikzpicture}"
        txt+="\n \\tkzTabInit{$"+latex(var)+"$ / 1 , $f("+latex(var)+")$ / 1}{"
        L=liste_x_signe(f,a,b)
        for i in L:
            txt+="$"+signe_infini(latex(i))+"$,"
        txt=txt[:-1]+"}"
        txt+="\n \\tkzTabLine{,"
        for i in range(len(L[:-1])):
            txt+=signe_f(f,L[i],L[i+1])+","
            if i!=len(L[:-1])-1:
                if est_val_interdite(f,L[i+1]):
                    txt+="d,"
                else:
                    txt+="z,"
        txt+="}\\end{tikzpicture}"
        return formate(txt)

def ecrire_sys(gauche1,droite1,gauche2,droite2):

    txt="$\\left \\{\\begin{matrix}"
    txt+=latex(gauche1)+" = "+latex(droite1)+"\\\\"
    txt+=latex(gauche2)+" = "+latex(droite2)
    txt+="\\end{matrix}\\right.$"
    return txt

def systeme_combi(x,y,a1,b1,c1,a2,b2,c2):
    x=Symbol(x)
    y=Symbol(y)
    gauche1,droite1=sympify(a1*x+b1*y),c1
    gauche2,droite2=sympify(a2*x+b2*y),c2
    txt=ecrire_sys(gauche1,droite1,gauche2,droite2)
    a,b=abs(lcm(a1,a2)),abs(lcm(b1,b2))
    if abs(a1)!=abs(a2) and abs(b1)!=abs(b2):
        if a==a1 or a==a2:
            d1,d2=abs(int(a/a1)),abs(int(a/a2))
        else:
            d1,d2=abs(int(b/b1)),abs(int(b/b2))

        gauche1,droite1=d1*gauche1,d1*droite1
        gauche2,droite2=d2*gauche2,d2*droite2
        txt+="\\\\ $ \\Leftrightarrow $ "+ecrire_sys(gauche1,droite1,gauche2,droite2)

    if a1*b2-a2*b1==0:
        if a1*c2-c1*a2==0:
            txt+="Le système admet une infinité de solutions"
        else:
            txt+="Le système n'admet pas de solution"

    else:
        if LC(gauche1,x)==LC(gauche2,x) or LC(gauche1,y)==LC(gauche2,y):
            gauche2,droite2=gauche1-gauche2,droite1-droite2
        else:
            gauche2,droite2=gauche1+gauche2,droite1+droite2
        txt+="\\\\ $ \\Leftrightarrow $ "+ecrire_sys(gauche1,droite1,gauche2,droite2)
        gauche2,droite2=gauche2/LC(gauche2),Rational(droite2,LC(gauche2))
        txt+="\\\\ $ \\Leftrightarrow $ "+ecrire_sys(gauche1,droite1,gauche2,droite2)
        if LC(gauche2,x)==1:
            gauche1=gauche1.replace(x,droite2)
        else:
            gauche1=gauche1.replace(y,droite2)
        txt+="\\\\ $ \\Leftrightarrow $ "+ecrire_sys(gauche1,droite1,gauche2,droite2)
        droite1=Rational(droite1-gauche1.replace(LM(gauche1),0),LC(gauche1))
        gauche1=LM(gauche1)
        txt+="\\\\$ \\Leftrightarrow $ "+ecrire_sys(gauche1,droite1,gauche2,droite2)
        if LM(gauche1)==x:
            txt+="\\\\ $S=\\left\\lbrace \\left("+latex(droite1)+" ; "+latex(droite2)+"\\right) \\right\\rbrace$"
        else:
            txt+="\\\\$S=\\left\\lbrace \\left("+latex(droite2)+" ; "+latex(droite1)+"\\right) \\right\\rbrace$"
    return formate(txt)





def systeme_substi(x,y,a1,b1,c1,a2,b2,c2):
    x=Symbol(x)
    y=Symbol(y)
    gauche1,droite1=sympify(a1*x+b1*y),c1
    gauche2,droite2=sympify(a2*x+b2*y),c2
    txt=ecrire_sys(gauche1,droite1,gauche2,droite2)
    if a1==1:
        cas=1
        droite1=c1-b1*y
        gauche1=x
        gauche2=Add(b2*y,Mul(a2,c1-b1*y,evaluate=false),evaluate=false)
    elif b1==1:
        cas=2
        droite1=c1-a1*x
        gauche1=y
        gauche2=Add(a2*x,Mul(b2,c1-a1*x,evaluate=false),evaluate=false)
    elif a2==1:
        cas=3
        droite2=c2-b2*y
        gauche2=x
        gauche1=Add(b1*y,Mul(a1,c2-b2*y,evaluate=false),evaluate=false)
    elif b2==1:
        cas=4
        droite2=c2-a2*x
        gauche2=y
        gauche1=Add(a1*x,Mul(b1,c2-a2*x,evaluate=false),evaluate=false)

    else:
        cas=5
        if b1!=0:
            droite1=(c1-a1*x)/b1
            gauche1=y
            gauche2=Add(a2*x,Mul(b2,(c1-a1*x)/b1,evaluate=false),evaluate=false)
        else:
            droite1=c1/a1
            gauche1=x
            gauche2=a2*c1/a1+b1*y

    txt+="\\\\ $ \\Leftrightarrow $ "+ecrire_sys(gauche1,droite1,gauche2,droite2)
    gauche1bis=simplify(gauche1)
    gauche2bis=simplify(gauche2)
    if gauche1bis!=gauche1 or gauche2bis!=gauche2:
        txt+="\\\\ $ \\Leftrightarrow $ "+ecrire_sys(gauche1bis,droite1,gauche2bis,droite2)
    if degree(gauche1bis)!=1:
        if gauche1bis==droite1:
            txt+="Le système admet une infinité de solutions"
        else:
            txt+="Le système n'admet pas de solutions"
    elif degree(gauche2bis)!=1:
        if gauche2bis==droite2:
            txt+="Le système admet une infinité de solutions"
        else:
            txt+="Le système n'admet pas de solutions"
    else:
        if cas==1:
            gauche1=gauche1bis
            droite2=solve(gauche2bis-droite2,y)[0]
            gauche2=y
            txt+="\\\\ $ \\Leftrightarrow $ "+ecrire_sys(gauche1,droite1,gauche2,droite2)
            droite1=droite1.replace(y,droite2)
            txt+="\\\\ $ \\Leftrightarrow $ "+ecrire_sys(gauche1,droite1,gauche2,droite2)
            txt+="\\\\ $S=\\left\\lbrace \\left("+latex(droite1)+" ; "+latex(droite2)+" \\right) \\right\\rbrace$"
        if cas==2 or cas==5:
            gauche1=gauche1bis
            droite2=solve(gauche2bis-droite2,x)[0]
            gauche2=x
            txt+="\\\\ $ \\Leftrightarrow $ "+ecrire_sys(gauche1,droite1,gauche2,droite2)
            droite1=droite1.replace(x,droite2)
            txt+="\\\\ $ \\Leftrightarrow $ "+ecrire_sys(gauche1,droite1,gauche2,droite2)
            txt+="\\\\ $S=\\left\\lbrace \\left("+latex(droite2)+" ; "+latex(droite1)+" \\right) \\right\\rbrace$"
        if cas==3:
            gauche2=gauche2bis
            droite1=solve(gauche1bis-droite1,y)[0]
            gauche1=y
            txt+="\\\\ $ \\Leftrightarrow $ "+ecrire_sys(gauche1,droite1,gauche2,droite2)
            droite2=droite2.replace(y,droite1)
            txt+="\\\\ $ \\Leftrightarrow $ "+ecrire_sys(gauche1,droite1,gauche2,droite2)
            txt+="\\\\ $S=\\left\\lbrace\\left ("+latex(droite2)+" ; "+latex(droite1)+" \\right) \\right\\rbrace$"
        if cas==4:
            gauche2=gauche2bis
            droite1=solve(gauche1bis-droite1,x)[0]
            gauche1=x
            txt+="\\\\ $ \\Leftrightarrow $ "+ecrire_sys(gauche1,droite1,gauche2,droite2)
            droite2=droite2.replace(x,droite1)
            txt+="\\\\ $ \\Leftrightarrow $ "+ecrire_sys(gauche1,droite1,gauche2,droite2)
            txt+="\\\\ $S=\\left\\lbrace \\left("+latex(droite1)+" ; "+latex(droite2)+" \\right) \\right\\rbrace$"
    return formate(txt)        
                

def coefficients(p):
    c=p.subs(x,0)
    p=expand((p-c)/x)
    b=p.subs(x,0)
    p=expand((p-b)/x)
    a= p.subs(x,0)
    return a,b,c

def est_un_carre(x):
    y=sqrt(x)
    return y==int(y)

def eq_trinome(p,ensemble):
    p=sympify(saisie_fonction(p))
    a,b,c=coefficients(p)
    d=b**2-4*a*c
    txt="$\\Delta=b^2-4ac="+parenthese(b)+"^2-"+"4\\times"+parenthese(a)+" \\times"+parenthese(c)+"="+latex(d)+"$\\\\"
    if ensemble=="R":
        if discriminant(p)<0:
            txt+="$\\Delta<0$ donc l'équation $"+latex(p)+"=0 $"+ " n'admet pas de solution \\\\"
        elif discriminant(p)==0:
            txt+="$\\Delta=0$ donc l'équation $ "+latex(p)+ "=0$ admet une unique solution : \\\\"
            txt+="$x_0=\\frac{-b}{2a}=\\dfrac{"+latex(-b)+"}{"+latex(2*a)+"}="+latex(-b/(2*a))+" \\\\$"
        else:
            txt+="$\\Delta>0 $ donc l'équation $"+latex(p)+ "=0$ admet deux solutions réelles :\\\\"
            txt+="$x_1=\\frac{-b-\sqrt{\\Delta}}{2a}=\\dfrac{-"+parenthese(b)+"-\\sqrt{"+latex(d)+"}"+"}{2\\times"+parenthese(a)+"}="
            txt+="\\frac{"+latex(-b-sqrt(d))+"}{"+latex(2*a)+"}="
            txt+=latex((-b-sqrt(d))/(2*a))+"$\\\\"
            txt+="$x_2=\\frac{-b+\sqrt{\\Delta}}{2a}=\\dfrac{-"+parenthese(b)+"+\\sqrt{"+latex(d)+"}"+"}{2\\times"+parenthese(a)+"}="
            txt+="\\frac{"+latex(-b+sqrt(d))+"}{"+latex(2*a)+"}="
            txt+=latex((-b+sqrt(d))/(2*a))+"$\\\\"
    else:
        if discriminant(p)<0:
             txt+="$\\Delta<0$ donc l'équation $"+latex(p.subs(x,z))+"=0$ "+ " admet deux solutions complexes conjuguées :\\\\"
             txt+="$z_1=\\frac{-b-{i}\\sqrt{\\Delta}}{2a}=\\dfrac{-"+parenthese(b)+"-{i}\\sqrt{"+str(-d)+"}}"+"{2\\times"+parenthese(a)+"}="
             txt+=latex((-b-sqrt(d))/(2*a))+"$\\\\"
             txt+="et $z_2=\\overline{z_1}="
             txt+=latex((-b+sqrt(d))/(2*a))+"$\\\\"
        elif discriminant(p)==0:
            txt+="$\\Delta=0$ donc l'équation $ "+latex(p.subs(x,z))+ "=0$ admet une unique solution : \\\\"
            txt+="$z_0=\\frac{-b}{2a}=\\dfrac{"+latex(-b)+"}{"+latex(2*a)+"}="+latex(-b/(2*a))+" \\\\$"
        else:
            txt+="$\\Delta>0 $ donc l'équation $"+latex(p.subs(x,z))+ "=0$ admet deux solutions réelles :\\\\"
            txt+="$z_1=\\frac{-b+\sqrt{\\Delta}}{2a}=\\dfrac{-"+parenthese(b)+"+\\sqrt{"+latex(d)+"}"+"}{2\\times"+parenthese(a)+"}="
            txt+="\\frac{"+latex(-b+sqrt(d))+"}{"+latex(2*a)+"}="
            txt+=latex((-b+sqrt(d))/(2*a))+"$\\\\"
            txt+="$z_2=\\frac{-b-\sqrt{\\Delta}}{2a}=\\dfrac{-"+parenthese(b)+"-\\sqrt{"+latex(d)+"}"+"}{2\\times"+parenthese(a)+"}="
            txt+="\\frac{"+latex(-b-sqrt(d))+"}{"+latex(2*a)+"}="
            txt+=latex((-b-sqrt(d))/(2*a))+"$\\\\"
    return formate(txt)	    

	
\end{pycode}
\begin{pycode}
def borne_intervalle(i):
    t=[]
    if i.is_EmptySet:
        return t
    else:
        if i.inf==-oo:
            t.append(-oo)
        for n in i.boundary:
            t.append(n)
        if i.sup==oo:
            t.append(+oo)
        return t

def liste_x_signe(f,a,b):
    var=variable(f)
    f=factor(sympify(f))
    bornes_plus=borne_intervalle(solve_univariate_inequality(f>0,var, relational=False).intersect(Interval(a,b)))
    bornes_moins=borne_intervalle(solve_univariate_inequality(f < 0, var, relational=False).intersect(Interval(a,b)))
    tab=list(set(bornes_plus+bornes_moins))
    return(sorted(tab))


def signe_f(f,a,b):
    var=variable(f)
    f=factor(sympify(f))
    I=solve_univariate_inequality(f>0,var, relational=False).intersect(Interval(a,b))
    if I.is_EmptySet:
        return "-"
    else:
        return "+"


def est_val_interdite(f,a):
    var=variable(f)
    f=sympify(f)
    bool=True
    if a==oo or a==-oo:
        image=limit(f,var,a)
    else:
        image=f.subs(var,a)
    try :
        if image!=zoo:
            bool=False
    except:
        if  image.is_real :
            bool= False
    return bool


def tabsigne(f,a,b):
        var=variable(f)
        f=saisie_fonction(f)
        f1=sympify(f)
        txt="\n \\begin{tikzpicture}"
        txt+="\n \\tkzTabInit{$"+latex(var)+"$ / 1 , $f("+latex(var)+")$ / 1}{"
        L=liste_x_signe(f,a,b)
        for i in L:
            txt+="$"+signe_infini(latex(i))+"$,"
        txt=txt[:-1]+"}"
        txt+="\n \\tkzTabLine{,"
        for i in range(len(L[:-1])):
            txt+=signe_f(f,L[i],L[i+1])+","
            if i!=len(L[:-1])-1:
                if est_val_interdite(f,L[i+1]):
                    txt+="d,"
                else:
                    txt+="z,"
        txt+="}\\end{tikzpicture}"
        return formate(txt)

\end{pycode}
\maketitle
\enableopenany
\tableofcontents


\chapter{Algèbre}
Pour pouvoir utiliser ces fonctions, il faut avoir écrire \pyv{\begin{pycode}
from sympy import *
x, y, z, t ,h,q,a,b,c,d = symbols('x y z t h q a b c d')

def saisie_fonction(txt):
    liste=['x','a','b','y','z','q','e','i','ln','n','\pi','t','sqrt','(']
    liste2=["a","b",'y','x','n','t','q','z']

    for i in range(0,10):
        for lettre in liste:
            txt=txt.replace(str(i)+lettre,str(i)+"*"+lettre)

    txt=txt.replace(")(",")*(")
    txt=txt.replace("ln","log")
    for lettre in liste2:
        if lettre!="t" or "sqrt" not in txt:
            txt=txt.replace(lettre+"(",lettre+"*(")
        txt=txt.replace(")"+lettre,")*"+lettre)
        txt=txt.replace("e^"+lettre,"exp("+lettre+")")
    txt=txt.replace("i","I")
    txt=txt.replace("e^","exp")
    txt=txt.replace("^","**")
    txt=txt.replace("{","(")
    txt=txt.replace("}",")")
    txt=txt.replace(" ","")
    txt=txt.replace(")sqrt",")*sqrt")
    txt=txt.replace(")e",")*e")
    txt=txt.replace(")log",")*log")
    txt=txt.replace("over","/")

    return (txt)


def formate(txt):
    txt=txt.replace("\\frac","\\dfrac")
    if not("left" in txt):
        txt=txt.replace("(","\\left(")
    if not "right" in txt:
        txt=txt.replace(")","\\right)")
    txt=txt.replace("\\equiv","\\Leftrightarrow")
    txt=txt.replace("\\infty","+\\infty")
    txt=txt.replace("log",'ln')
    txt=txt.replace("--","+")
    txt=txt.replace("+-","-")
    txt=txt.replace("-+","-")
    txt=txt.replace("++","+")
    txt=txt.replace("\\vec","\\overrightarrow")
    txt=txt.replace(">="," \\ge ")
    txt=txt.replace("<="," \\le ")
    return txt

def change_signe(txt):
    txt=txt.replace("+","plus")
    txt=txt.replace("-","+")
    txt=txt.replace("plus","-")
    return txt

def signe_infini(a):
    if a=="\\infty":
        return "+"+a
    else:
        return a


def signeplus(fonction):
    func=latex(fonction)
    if func[0]!="-":
        return "+"
    else:
        return ""

def ajouteplus(nbre):
    if nbre>=0:
        return "+"+latex(nbre)
    else:
        return latex(nbre)
        
def parenthese(nbre):
    if nbre<0:
        return "("+latex(nbre)+")"
    else:
        return latex(nbre)

def variable(fonction):
    var=Symbol("x")
    if ("n" in fonction) and ("ln" not in fonction):
        var=Symbol("n")
    if "t" in fonction and "sqrt" not in fonction:
        var=Symbol("t")
    if "q" in fonction and "sqrt" not in fonction:
        var=Symbol("q")
    if "p" in fonction and "exp" not in fonction:
        var=Symbol("p")
    return var

def str2list(txt):
    L1=txt.split()
    L2=[]
    for elt in L1:
        L2.append(float(elt))
    return L2


\end{pycode}
} et \pyv{\begin{pycode}
x, y, z, t ,h,q,a,b,c= symbols('x y z t h q a b c')


def equation_degre1(gauche,droite,symbole):
    gauche=saisie_fonction(gauche)
    droite=saisie_fonction(droite)
    var=variable(gauche)
    g=sympify(gauche)
    d=sympify(droite)
    if symbole !="=":
        ineq=sympify(gauche+symbole+droite)

    if (degree(d-g,var)>1) or degree(d-g,var)==0:
        txt="Ce n'est pas une équation ou une inéquation du premier degré"
    else:
        txt="$"+latex(g)+symbole+latex(d)+"$\\\\"
        g1=sympify(gauche)
        d1=sympify(droite)
        if (latex(g)!=latex(g1) or latex(d)!=latex(d1)):
            txt+="$ \\Leftrightarrow "+latex(g1)+symbole+latex(d1)+"$\\\\"

        g2=g1-g1.replace(var,0)
        d2=d1
        if degree(d,var)==1:
            d2=d1-LT(d1)

        if degree(d,var)!=0 or g.replace(var,0)!=0:
            txt+="$ \\Leftrightarrow "+latex(g2)
            if degree(d,var)==1:
                txt+="-"+latex(LT(d1))
            txt+=symbole+latex(d2)
            if g1.replace(var,0)!=0:
                txt+="-"+latex(g1.replace(var,0))
            txt+="$\\\\"

            if degree(d,var)==1:
                g2=g2-LT(d1)
            d2=d2-g1.replace(var,0)
            txt+="$ \\Leftrightarrow "+latex(g2)+symbole+latex(d2)+"$\\\\"
        if degree(g2,var)!=1:
            if symbole=="=":
                if d2==0:
                    txt1="L'ensemble solution est : $S=\\mathbb{R}$"
                else:
                    txt1="L'ensemble solution est : $S=\\emptyset$"
            else:
                txt1="L'ensemble solution est : $S="+latex(solve_univariate_inequality(ineq,var,relational=False))+"$"
        else:
            if LC(g2)<0:
                if ">" in symbole:
                    symbole=symbole.replace(">","<")
                else:
                    symbole=symbole.replace("<",">")
            if LC(g-d)!=1:
                txt+="$ \\Leftrightarrow "+latex(g2/LC(g2))+symbole+"\\dfrac{"+latex(d2)+"}{"+latex(LC(g2))+"}$\\\\"
                if latex(d2/LC(g2))!="\\frac{"+latex(d2)+"}{"+latex(LC(g2))+"}":
                    txt+="$ \\Leftrightarrow "+latex(g2/LC(g2))+symbole+latex(d2/LC(g2))+"$\\\\"
            if symbole=="=":
                txt1="L'ensemble solution est S=$\\left\\{"+latex(d2/LC(g2))+"\\right\\}$"
            else:

                sol=latex(solve_univariate_inequality(ineq,var,relational=False))
                sol=sol.replace("(","]")
                sol=sol.replace(")","[")
                txt1="L'ensemble solution est S=$"+sol+"$"


    return(formate(txt),formate(txt1))


def tableau_signe_produit(f1,f2):
    f1,f2=sympify(saisie_fonction(f1)),sympify(saisie_fonction(f2))
    a1,b1=LC(f1),f1.replace(x,0)
    a2,b2=LC(f2),f2.replace(x,0)
    alpha1,alpha2=-b1/a1,-b2/a2
    ligne0="$-\\infty$, $"+latex(alpha1)+"$, $"+latex(alpha2)+"$, $+\\infty$"
    ligne1=" ,-,z,+,t,+, "
    ligne2=" ,-,t,-,z,+,"
    ligne3=" ,+,z,-,z,+, "
    if alpha1>alpha2:
        ligne0="$-\\infty$, $"+latex(alpha2)+"$, $"+latex(alpha1)+"$, $+\\infty$"
        ligne1,ligne2=ligne2,ligne1
    if a1<0:
        ligne1=change_signe(ligne1)
        ligne3=change_signe(ligne3)
    if a2<0:
        ligne2=change_signe(ligne2)
        ligne3=change_signe(ligne3)

    f1=latex(a1)+"x+"+latex(b1)
    f2=latex(a2)+"x+"+latex(b2)
    f1f2="("+f1+")("+f2+")"
    txt="\\resizebox{.7\\textwidth}{!}{"
    txt+="\\begin{tikzpicture}"
    txt+="\\tkzTabInit[lgt=5]{$x$ / 1 , $"+f1+"$ / 1,$"+f2+"$ / 1,$"+f1f2+"$ / 1 }{"+ligne0+"}"
    txt+="\\tkzTabLine{"+ligne1+"}"
    txt+="\\tkzTabLine{"+ligne2+"}"
    txt+="\\tkzTabLine{"+ligne3+"}"
    txt+="\\end{tikzpicture}}"
    return formate(txt)

def tableau_signe_quotient(f1,f2):
    f1,f2=sympify(saisie_fonction(f1)),sympify(saisie_fonction(f2))
    a1,b1=LC(f1),f1.replace(x,0)
    a2,b2=LC(f2),f2.replace(x,0)
    alpha1,alpha2=-b1/a1,-b2/a2
    ligne0="$-\\infty$, $"+latex(alpha1)+"$, $"+latex(alpha2)+"$, $+\\infty$"
    ligne1=" ,-,z,+,t,+, "
    ligne2=" ,-,t,-,z,+,"
    ligne3=" ,+,z,-,d,+,"
    if alpha1>alpha2:
        ligne0="$-\\infty$, $"+latex(alpha2)+"$, $"+latex(alpha1)+"$, $+\\infty$"
        ligne1,ligne2=ligne2,ligne1
        ligne3=" ,+,d,-,z,+, "

    if a1<0:
        ligne1=change_signe(ligne1)
        ligne3=change_signe(ligne3)
    if a2<0:
        ligne2=change_signe(ligne2)
        ligne3=change_signe(ligne3)
    f1=latex(a1)+"x+"+latex(b1)
    f2=latex(a2)+"x+"+latex(b2)
    f1f2="\\dfrac{"+f1+"}{"+f2+"}"
    txt="\\resizebox{.7\\textwidth}{!}{"
    txt+="\\begin{tikzpicture}"
    txt+="\\tkzTabInit[lgt=3]{$x$ / 1 , $"+f1+"$ / 1,$"+f2+"$ / 1,$"+f1f2+"$ / 1 }{"+ligne0+"}"
    txt+="\\tkzTabLine{"+ligne1+"}"
    txt+="\\tkzTabLine{"+ligne2+"}"
    txt+="\\tkzTabLine{"+ligne3+"}"
    txt+="\\end{tikzpicture}}"
    return formate(txt)

def ineq_produit(f1,f2,symbole):
    f1,f2=saisie_fonction(f1),saisie_fonction(f2)
    txt=equation_degre1(f1,"0",'>')[0].replace("\\\\","")+"\\\\"
    txt+=equation_degre1(f2,"0",'>')[0].replace("\\\\","")+"\\\\"
    txt+="On obtient ainsi le tableau suivant : \\\\"
    txt+=tableau_signe_produit(f1,f2)+"\\\\"
    f="("+latex(f1)+")*("+latex(f2)+")"
    txt+=sol_ineq(f+symbole+"0")[0]+sol_ineq(f+symbole+"0")[1]
    return txt

def ineq_quotient(f1,f2,symbole):
    f1,f2=saisie_fonction(f1),saisie_fonction(f2)
    txt=equation_degre1(f1,"0",'>')[0].replace("\\\\","")+"\\\\"
    txt+=equation_degre1(f2,"0",'>')[0].replace("\\\\","")+"\\\\"
    txt+="On obtient ainsi le tableau suivant : \\\\"
    txt+=tableau_signe_quotient(f1,f2)+"\\\\"
    f="("+latex(f1)+")/("+latex(f2)+")"
    txt+=sol_ineq(f+symbole+"0")[0]+sol_ineq(f+symbole+"0")[1]
    return txt

def sol_ineq(ineq):
    ineq=sympify(saisie_fonction(ineq))
    txt1="L'ensemble solution de $"+latex(ineq)+"$ est "
    txt="$"+latex(solve_univariate_inequality(ineq,x,relational=False))+"$"
    txt=txt.replace("(","]")
    txt=txt.replace(")","[")
    return formate(txt1),formate(txt)

def borne_intervalle(i):
    t=[]
    if i.is_EmptySet:
        return t
    else:
        if i.inf==-oo:
            t.append(-oo)
        for n in i.boundary:
            t.append(n)
        if i.sup==oo:
            t.append(+oo)
        return t

def liste_x_signe(f,a,b):
    var=variable(f)
    f=factor(sympify(f))
    bornes_plus=borne_intervalle(solve_univariate_inequality(f>0,var, relational=False).intersect(Interval(a,b)))
    bornes_moins=borne_intervalle(solve_univariate_inequality(f < 0, var, relational=False).intersect(Interval(a,b)))
    tab=list(set(bornes_plus+bornes_moins))
    return(sorted(tab))


def signe_f(f,a,b):
    var=variable(f)
    f=factor(sympify(f))
    I=solve_univariate_inequality(f>0,var, relational=False).intersect(Interval(a,b))
    if I.is_EmptySet:
        return "-"
    else:
        return "+"


def est_val_interdite(f,a):
    var=variable(f)
    f=sympify(f)
    bool=True
    if a==oo or a==-oo:
        image=limit(f,var,a)
    else:
        image=f.subs(var,a)
    try :
        if image!=zoo:
            bool=False
    except:
        if  image.is_real :
            bool= False
    return bool


def tabsigne(f,a,b):
        var=variable(f)
        f=saisie_fonction(f)
        f1=sympify(f)
        txt="\n \\begin{tikzpicture}"
        txt+="\n \\tkzTabInit{$"+latex(var)+"$ / 1 , $f("+latex(var)+")$ / 1}{"
        L=liste_x_signe(f,a,b)
        for i in L:
            txt+="$"+signe_infini(latex(i))+"$,"
        txt=txt[:-1]+"}"
        txt+="\n \\tkzTabLine{,"
        for i in range(len(L[:-1])):
            txt+=signe_f(f,L[i],L[i+1])+","
            if i!=len(L[:-1])-1:
                if est_val_interdite(f,L[i+1]):
                    txt+="d,"
                else:
                    txt+="z,"
        txt+="}\\end{tikzpicture}"
        return formate(txt)

def ecrire_sys(gauche1,droite1,gauche2,droite2):

    txt="$\\left \\{\\begin{matrix}"
    txt+=latex(gauche1)+" = "+latex(droite1)+"\\\\"
    txt+=latex(gauche2)+" = "+latex(droite2)
    txt+="\\end{matrix}\\right.$"
    return txt

def systeme_combi(x,y,a1,b1,c1,a2,b2,c2):
    x=Symbol(x)
    y=Symbol(y)
    gauche1,droite1=sympify(a1*x+b1*y),c1
    gauche2,droite2=sympify(a2*x+b2*y),c2
    txt=ecrire_sys(gauche1,droite1,gauche2,droite2)
    a,b=abs(lcm(a1,a2)),abs(lcm(b1,b2))
    if abs(a1)!=abs(a2) and abs(b1)!=abs(b2):
        if a==a1 or a==a2:
            d1,d2=abs(int(a/a1)),abs(int(a/a2))
        else:
            d1,d2=abs(int(b/b1)),abs(int(b/b2))

        gauche1,droite1=d1*gauche1,d1*droite1
        gauche2,droite2=d2*gauche2,d2*droite2
        txt+="\\\\ $ \\Leftrightarrow $ "+ecrire_sys(gauche1,droite1,gauche2,droite2)

    if a1*b2-a2*b1==0:
        if a1*c2-c1*a2==0:
            txt+="Le système admet une infinité de solutions"
        else:
            txt+="Le système n'admet pas de solution"

    else:
        if LC(gauche1,x)==LC(gauche2,x) or LC(gauche1,y)==LC(gauche2,y):
            gauche2,droite2=gauche1-gauche2,droite1-droite2
        else:
            gauche2,droite2=gauche1+gauche2,droite1+droite2
        txt+="\\\\ $ \\Leftrightarrow $ "+ecrire_sys(gauche1,droite1,gauche2,droite2)
        gauche2,droite2=gauche2/LC(gauche2),Rational(droite2,LC(gauche2))
        txt+="\\\\ $ \\Leftrightarrow $ "+ecrire_sys(gauche1,droite1,gauche2,droite2)
        if LC(gauche2,x)==1:
            gauche1=gauche1.replace(x,droite2)
        else:
            gauche1=gauche1.replace(y,droite2)
        txt+="\\\\ $ \\Leftrightarrow $ "+ecrire_sys(gauche1,droite1,gauche2,droite2)
        droite1=Rational(droite1-gauche1.replace(LM(gauche1),0),LC(gauche1))
        gauche1=LM(gauche1)
        txt+="\\\\$ \\Leftrightarrow $ "+ecrire_sys(gauche1,droite1,gauche2,droite2)
        if LM(gauche1)==x:
            txt+="\\\\ $S=\\left\\lbrace \\left("+latex(droite1)+" ; "+latex(droite2)+"\\right) \\right\\rbrace$"
        else:
            txt+="\\\\$S=\\left\\lbrace \\left("+latex(droite2)+" ; "+latex(droite1)+"\\right) \\right\\rbrace$"
    return formate(txt)





def systeme_substi(x,y,a1,b1,c1,a2,b2,c2):
    x=Symbol(x)
    y=Symbol(y)
    gauche1,droite1=sympify(a1*x+b1*y),c1
    gauche2,droite2=sympify(a2*x+b2*y),c2
    txt=ecrire_sys(gauche1,droite1,gauche2,droite2)
    if a1==1:
        cas=1
        droite1=c1-b1*y
        gauche1=x
        gauche2=Add(b2*y,Mul(a2,c1-b1*y,evaluate=false),evaluate=false)
    elif b1==1:
        cas=2
        droite1=c1-a1*x
        gauche1=y
        gauche2=Add(a2*x,Mul(b2,c1-a1*x,evaluate=false),evaluate=false)
    elif a2==1:
        cas=3
        droite2=c2-b2*y
        gauche2=x
        gauche1=Add(b1*y,Mul(a1,c2-b2*y,evaluate=false),evaluate=false)
    elif b2==1:
        cas=4
        droite2=c2-a2*x
        gauche2=y
        gauche1=Add(a1*x,Mul(b1,c2-a2*x,evaluate=false),evaluate=false)

    else:
        cas=5
        if b1!=0:
            droite1=(c1-a1*x)/b1
            gauche1=y
            gauche2=Add(a2*x,Mul(b2,(c1-a1*x)/b1,evaluate=false),evaluate=false)
        else:
            droite1=c1/a1
            gauche1=x
            gauche2=a2*c1/a1+b1*y

    txt+="\\\\ $ \\Leftrightarrow $ "+ecrire_sys(gauche1,droite1,gauche2,droite2)
    gauche1bis=simplify(gauche1)
    gauche2bis=simplify(gauche2)
    if gauche1bis!=gauche1 or gauche2bis!=gauche2:
        txt+="\\\\ $ \\Leftrightarrow $ "+ecrire_sys(gauche1bis,droite1,gauche2bis,droite2)
    if degree(gauche1bis)!=1:
        if gauche1bis==droite1:
            txt+="Le système admet une infinité de solutions"
        else:
            txt+="Le système n'admet pas de solutions"
    elif degree(gauche2bis)!=1:
        if gauche2bis==droite2:
            txt+="Le système admet une infinité de solutions"
        else:
            txt+="Le système n'admet pas de solutions"
    else:
        if cas==1:
            gauche1=gauche1bis
            droite2=solve(gauche2bis-droite2,y)[0]
            gauche2=y
            txt+="\\\\ $ \\Leftrightarrow $ "+ecrire_sys(gauche1,droite1,gauche2,droite2)
            droite1=droite1.replace(y,droite2)
            txt+="\\\\ $ \\Leftrightarrow $ "+ecrire_sys(gauche1,droite1,gauche2,droite2)
            txt+="\\\\ $S=\\left\\lbrace \\left("+latex(droite1)+" ; "+latex(droite2)+" \\right) \\right\\rbrace$"
        if cas==2 or cas==5:
            gauche1=gauche1bis
            droite2=solve(gauche2bis-droite2,x)[0]
            gauche2=x
            txt+="\\\\ $ \\Leftrightarrow $ "+ecrire_sys(gauche1,droite1,gauche2,droite2)
            droite1=droite1.replace(x,droite2)
            txt+="\\\\ $ \\Leftrightarrow $ "+ecrire_sys(gauche1,droite1,gauche2,droite2)
            txt+="\\\\ $S=\\left\\lbrace \\left("+latex(droite2)+" ; "+latex(droite1)+" \\right) \\right\\rbrace$"
        if cas==3:
            gauche2=gauche2bis
            droite1=solve(gauche1bis-droite1,y)[0]
            gauche1=y
            txt+="\\\\ $ \\Leftrightarrow $ "+ecrire_sys(gauche1,droite1,gauche2,droite2)
            droite2=droite2.replace(y,droite1)
            txt+="\\\\ $ \\Leftrightarrow $ "+ecrire_sys(gauche1,droite1,gauche2,droite2)
            txt+="\\\\ $S=\\left\\lbrace\\left ("+latex(droite2)+" ; "+latex(droite1)+" \\right) \\right\\rbrace$"
        if cas==4:
            gauche2=gauche2bis
            droite1=solve(gauche1bis-droite1,x)[0]
            gauche1=x
            txt+="\\\\ $ \\Leftrightarrow $ "+ecrire_sys(gauche1,droite1,gauche2,droite2)
            droite2=droite2.replace(x,droite1)
            txt+="\\\\ $ \\Leftrightarrow $ "+ecrire_sys(gauche1,droite1,gauche2,droite2)
            txt+="\\\\ $S=\\left\\lbrace \\left("+latex(droite1)+" ; "+latex(droite2)+" \\right) \\right\\rbrace$"
    return formate(txt)        
                

def coefficients(p):
    c=p.subs(x,0)
    p=expand((p-c)/x)
    b=p.subs(x,0)
    p=expand((p-b)/x)
    a= p.subs(x,0)
    return a,b,c

def est_un_carre(x):
    y=sqrt(x)
    return y==int(y)

def eq_trinome(p,ensemble):
    p=sympify(saisie_fonction(p))
    a,b,c=coefficients(p)
    d=b**2-4*a*c
    txt="$\\Delta=b^2-4ac="+parenthese(b)+"^2-"+"4\\times"+parenthese(a)+" \\times"+parenthese(c)+"="+latex(d)+"$\\\\"
    if ensemble=="R":
        if discriminant(p)<0:
            txt+="$\\Delta<0$ donc l'équation $"+latex(p)+"=0 $"+ " n'admet pas de solution \\\\"
        elif discriminant(p)==0:
            txt+="$\\Delta=0$ donc l'équation $ "+latex(p)+ "=0$ admet une unique solution : \\\\"
            txt+="$x_0=\\frac{-b}{2a}=\\dfrac{"+latex(-b)+"}{"+latex(2*a)+"}="+latex(-b/(2*a))+" \\\\$"
        else:
            txt+="$\\Delta>0 $ donc l'équation $"+latex(p)+ "=0$ admet deux solutions réelles :\\\\"
            txt+="$x_1=\\frac{-b-\sqrt{\\Delta}}{2a}=\\dfrac{-"+parenthese(b)+"-\\sqrt{"+latex(d)+"}"+"}{2\\times"+parenthese(a)+"}="
            txt+="\\frac{"+latex(-b-sqrt(d))+"}{"+latex(2*a)+"}="
            txt+=latex((-b-sqrt(d))/(2*a))+"$\\\\"
            txt+="$x_2=\\frac{-b+\sqrt{\\Delta}}{2a}=\\dfrac{-"+parenthese(b)+"+\\sqrt{"+latex(d)+"}"+"}{2\\times"+parenthese(a)+"}="
            txt+="\\frac{"+latex(-b+sqrt(d))+"}{"+latex(2*a)+"}="
            txt+=latex((-b+sqrt(d))/(2*a))+"$\\\\"
    else:
        if discriminant(p)<0:
             txt+="$\\Delta<0$ donc l'équation $"+latex(p.subs(x,z))+"=0$ "+ " admet deux solutions complexes conjuguées :\\\\"
             txt+="$z_1=\\frac{-b-{i}\\sqrt{\\Delta}}{2a}=\\dfrac{-"+parenthese(b)+"-{i}\\sqrt{"+str(-d)+"}}"+"{2\\times"+parenthese(a)+"}="
             txt+=latex((-b-sqrt(d))/(2*a))+"$\\\\"
             txt+="et $z_2=\\overline{z_1}="
             txt+=latex((-b+sqrt(d))/(2*a))+"$\\\\"
        elif discriminant(p)==0:
            txt+="$\\Delta=0$ donc l'équation $ "+latex(p.subs(x,z))+ "=0$ admet une unique solution : \\\\"
            txt+="$z_0=\\frac{-b}{2a}=\\dfrac{"+latex(-b)+"}{"+latex(2*a)+"}="+latex(-b/(2*a))+" \\\\$"
        else:
            txt+="$\\Delta>0 $ donc l'équation $"+latex(p.subs(x,z))+ "=0$ admet deux solutions réelles :\\\\"
            txt+="$z_1=\\frac{-b+\sqrt{\\Delta}}{2a}=\\dfrac{-"+parenthese(b)+"+\\sqrt{"+latex(d)+"}"+"}{2\\times"+parenthese(a)+"}="
            txt+="\\frac{"+latex(-b+sqrt(d))+"}{"+latex(2*a)+"}="
            txt+=latex((-b+sqrt(d))/(2*a))+"$\\\\"
            txt+="$z_2=\\frac{-b-\sqrt{\\Delta}}{2a}=\\dfrac{-"+parenthese(b)+"-\\sqrt{"+latex(d)+"}"+"}{2\\times"+parenthese(a)+"}="
            txt+="\\frac{"+latex(-b-sqrt(d))+"}{"+latex(2*a)+"}="
            txt+=latex((-b-sqrt(d))/(2*a))+"$\\\\"
    return formate(txt)	    

	
\end{pycode}} après \pyv{\begin{document}}
\section{Équation et inéquation du 1er degré}
\pyv{equation_degre1(gauche,droite,symbole)} permet de résoudre des équations et inéquations du premier degré. La fonction renvoie une liste. Le premier élément correspond à la résolution et le deuxième à l'ensemble solution.
\sol{
\textbf{Exemple 1} : \pyv{equation_degre1("5x+3","x+2","=")[0]} et \pyv{[1]}
\begin{multicols*}{2}
\py{equation_degre1("5x+3","x+2","=")[0]}\newpage
\py{equation_degre1("5x+3","x+2","=")[1]}
\end{multicols*}
\textbf{Exemple 2} : \pyv{equation_degre1("1t+3","5t-1",">=")[0]} et \pyv{[1]}
\begin{multicols*}{2}
\py{equation_degre1("1t+3","5t-1",">=")[0]}\newpage
\py{equation_degre1("1t+3","5t-1",">=")[1]}
\end{multicols*}
}
\section{Ensemble solution d'une inéquation}
\pyv{sol_ineq(ineq)} renvoie une liste dont la première valeur est la phrase \textit{"l'ensemble solution de l'inéquation...est "} et la deuxième valeur est l'ensemble solution.
\sol{
\textbf{Exemple 1}:\pyv{sol_ineq("5x+1>x+2")[0]} et \pyv{[1]}\\
 \py{sol_ineq("5*x+1>x+2")[0]}\py{sol_ineq("5*x+1>x+2")[1]} \\ 
 \textbf{Exemple 2}:\pyv{sol_ineq("x^2+3x+1>x-2")[0]} et \pyv{[1]}\\
 \py{sol_ineq("x^2+3x+1>x+2")[0]}\py{sol_ineq("x^2+3x+1>x+2")[1]} 
 }
 
\section{Tableau de signe du produit de 2 fonctions affines}
\pyv{tableau_signe_produit(f1,f2)} permet de tracer le tableau de signe du produit de deux fonctions affines f1 et f2
\sol{
\textbf{Exemple :}  \pyv{ tableau_signe_produit("5x-2","-2x+3")}\\
\py{tableau_signe_produit("5x-2","-2x+3")}
}
\section{Tableau de signe du quotient de 2 fonctions affines}
\pyv{tableau_signe_quotient(f1,f2)} permet de tracer le tableau de signe du quotient de deux fonctions affines f1 et f2
\sol{
\textbf{Exemple : } \pyv{tableau_signe_quotient("5x-2","-2x+3")}\\
\py{tableau_signe_quotient("5x-2","-2x+3")}
}

\section{Inéquation produit de 2 fonctions affines}
\pyv{ineq_produit(f1,f2,symbole)} permet de résoudre les inéquation produit nul de deux fonctions affines à l'aide d'un tableau de signe.
\sol{
\textbf{Exemple : } \pyv{ineq_produit("5x-2","-2x+3",">")}\\
\py{ineq_produit("5x-2","-2x+3",">")}
}
\section{Inéquation quotient de 2 fonctions affines}
\pyv{ineq_quotient(f1,f2,symbole)} permet de résoudre les inéquation quotient nul de deux fonctions affines à l'aide d'un tableau de signe.
\sol{
\textbf{Exemple :}  \pyv{ineq_quotient("5x-2","-2x+3,"<=")}\\
\py{ineq_quotient("5x-2","-2x+3","<=")}
}

\section{Tableau de signe d'une fonction}
\pyv{tabsigne(f,a,b)} permet de tracer le tableau de signe de la fonction $f$ sur l'intervalle $]a;b[$ 
\sol{
\textbf{Exemple 1 :}  \pyv{tabsigne("e^x-2",-10,10)}\\
\py{tabsigne("e^x-2",-10,10)}\\
\textbf{Exemple 2 :}  \pyv{tabsigne("x^3",-oo,+oo)}\\
\py{tabsigne("x^3",-oo,+oo)}
}
\section{Résolution d'un système par substitution}
\pyv{systeme_substi(x,y,a1,b1,c1,a2,b2,c2)} permet de résoudre le système 
$\left \{\begin{matrix}a1.x+b1.y=c1 \\a2.x+b2.y=c2 \end{matrix}\right.$ en utilisant la méthode par substitution.
\sol{\textbf{Exemple :} \pyv{systeme_substi("a","b",2,3,4,-4,1,5)}\\
\begin{multicols}{3}
\py{systeme_substi("a","b",2,3,4,-4,1,5)}
\end{multicols}

}

\section{Résolution d'un système par combinaison linéaire}
\pyv{systeme_combi(x,y,a1,b1,c1,a2,b2,c2)} permet de résoudre le système en utilisant la méthode par combinaison linéaire.
$\left \{\begin{matrix}a1.x+b1.y=c1 \\a2.x+b2.y=c2 \end{matrix}\right.$
\sol{\textbf{Exemple :} \pyv{systeme_combi("x","y",2,3,4,-4,1,5)}\\
\begin{multicols}{3}
\py{systeme_combi("x","y",2,3,4,-4,1,5)}
\end{multicols}
}
\section{Résolution d'équation du second degré}
\pyv{eq_trinome(p,ensemble)} permet de résoudre une équation du second degré en calculant le discriminant. Si ensemble='R', on travaille dans $\mathbb{R}$ et si ensemble='C', on travaille dans $\mathbb{C}$
\sol{\textbf{Exemple 1 :} \pyv{eq_trinome("2*x^2+3x+1","R")}\\
\py{eq_trinome("2*x^2+3x+1","R")}\\
\textbf{Exemple 2 :}\pyv{eq_trinome("x^2+x+1","C")}\\
\py{eq_trinome("x^2+x+1","C")}
}
\chapter{automatex\_ana}


\end{document}
