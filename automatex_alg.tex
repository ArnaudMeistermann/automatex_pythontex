\begin{pycode}
x, y, z, t ,h,q,a,b,c= symbols('x y z t h q a b c')


def equation_degre1(gauche,droite,symbole):
    gauche=saisie_fonction(gauche)
    droite=saisie_fonction(droite)
    var=variable(gauche)
    g=sympify(gauche)
    d=sympify(droite)
    if symbole !="=":
        ineq=sympify(gauche+symbole+droite)

    if (degree(d-g,var)>1) or degree(d-g,var)==0:
        txt="Ce n'est pas une équation ou une inéquation du premier degré"
    else:
        txt="$"+latex(g)+symbole+latex(d)+"$\\\\"
        g1=sympify(gauche)
        d1=sympify(droite)
        if (latex(g)!=latex(g1) or latex(d)!=latex(d1)):
            txt+="$ \\Leftrightarrow "+latex(g1)+symbole+latex(d1)+"$\\\\"

        g2=g1-g1.replace(var,0)
        d2=d1
        if degree(d,var)==1:
            d2=d1-LT(d1)

        if degree(d,var)!=0 or g.replace(var,0)!=0:
            txt+="$ \\Leftrightarrow "+latex(g2)
            if degree(d,var)==1:
                txt+="-"+latex(LT(d1))
            txt+=symbole+latex(d2)
            if g1.replace(var,0)!=0:
                txt+="-"+latex(g1.replace(var,0))
            txt+="$\\\\"

            if degree(d,var)==1:
                g2=g2-LT(d1)
            d2=d2-g1.replace(var,0)
            txt+="$ \\Leftrightarrow "+latex(g2)+symbole+latex(d2)+"$\\\\"
        if degree(g2,var)!=1:
            if symbole=="=":
                if d2==0:
                    txt1="L'ensemble solution est : $S=\\mathbb{R}$"
                else:
                    txt1="L'ensemble solution est : $S=\\emptyset$"
            else:
                txt1="L'ensemble solution est : $S="+latex(solve_univariate_inequality(ineq,var,relational=False))+"$"
        else:
            if LC(g2)<0:
                if ">" in symbole:
                    symbole=symbole.replace(">","<")
                else:
                    symbole=symbole.replace("<",">")
            if LC(g-d)!=1:
                txt+="$ \\Leftrightarrow "+latex(g2/LC(g2))+symbole+"\\dfrac{"+latex(d2)+"}{"+latex(LC(g2))+"}$\\\\"
                if latex(d2/LC(g2))!="\\frac{"+latex(d2)+"}{"+latex(LC(g2))+"}":
                    txt+="$ \\Leftrightarrow "+latex(g2/LC(g2))+symbole+latex(d2/LC(g2))+"$\\\\"
            if symbole=="=":
                txt1="L'ensemble solution est S=$\\left\\{"+latex(d2/LC(g2))+"\\right\\}$"
            else:

                sol=latex(solve_univariate_inequality(ineq,var,relational=False))
                sol=sol.replace("(","]")
                sol=sol.replace(")","[")
                txt1="L'ensemble solution est S=$"+sol+"$"


    return(formate(txt),formate(txt1))


def tableau_signe_produit(f1,f2):
    f1,f2=sympify(saisie_fonction(f1)),sympify(saisie_fonction(f2))
    a1,b1=LC(f1),f1.replace(x,0)
    a2,b2=LC(f2),f2.replace(x,0)
    alpha1,alpha2=-b1/a1,-b2/a2
    ligne0="$-\\infty$, $"+latex(alpha1)+"$, $"+latex(alpha2)+"$, $+\\infty$"
    ligne1=" ,-,z,+,t,+, "
    ligne2=" ,-,t,-,z,+,"
    ligne3=" ,+,z,-,z,+, "
    if alpha1>alpha2:
        ligne0="$-\\infty$, $"+latex(alpha2)+"$, $"+latex(alpha1)+"$, $+\\infty$"
        ligne1,ligne2=ligne2,ligne1
    if a1<0:
        ligne1=change_signe(ligne1)
        ligne3=change_signe(ligne3)
    if a2<0:
        ligne2=change_signe(ligne2)
        ligne3=change_signe(ligne3)

    f1=latex(a1)+"x+"+latex(b1)
    f2=latex(a2)+"x+"+latex(b2)
    f1f2="("+f1+")("+f2+")"
    txt="\\resizebox{.7\\textwidth}{!}{"
    txt+="\\begin{tikzpicture}"
    txt+="\\tkzTabInit[lgt=5]{$x$ / 1 , $"+f1+"$ / 1,$"+f2+"$ / 1,$"+f1f2+"$ / 1 }{"+ligne0+"}"
    txt+="\\tkzTabLine{"+ligne1+"}"
    txt+="\\tkzTabLine{"+ligne2+"}"
    txt+="\\tkzTabLine{"+ligne3+"}"
    txt+="\\end{tikzpicture}}"
    return formate(txt)

def tableau_signe_quotient(f1,f2):
    f1,f2=sympify(saisie_fonction(f1)),sympify(saisie_fonction(f2))
    a1,b1=LC(f1),f1.replace(x,0)
    a2,b2=LC(f2),f2.replace(x,0)
    alpha1,alpha2=-b1/a1,-b2/a2
    ligne0="$-\\infty$, $"+latex(alpha1)+"$, $"+latex(alpha2)+"$, $+\\infty$"
    ligne1=" ,-,z,+,t,+, "
    ligne2=" ,-,t,-,z,+,"
    ligne3=" ,+,z,-,d,+,"
    if alpha1>alpha2:
        ligne0="$-\\infty$, $"+latex(alpha2)+"$, $"+latex(alpha1)+"$, $+\\infty$"
        ligne1,ligne2=ligne2,ligne1
        ligne3=" ,+,d,-,z,+, "

    if a1<0:
        ligne1=change_signe(ligne1)
        ligne3=change_signe(ligne3)
    if a2<0:
        ligne2=change_signe(ligne2)
        ligne3=change_signe(ligne3)
    f1=latex(a1)+"x+"+latex(b1)
    f2=latex(a2)+"x+"+latex(b2)
    f1f2="\\dfrac{"+f1+"}{"+f2+"}"
    txt="\\resizebox{.7\\textwidth}{!}{"
    txt+="\\begin{tikzpicture}"
    txt+="\\tkzTabInit[lgt=3]{$x$ / 1 , $"+f1+"$ / 1,$"+f2+"$ / 1,$"+f1f2+"$ / 1 }{"+ligne0+"}"
    txt+="\\tkzTabLine{"+ligne1+"}"
    txt+="\\tkzTabLine{"+ligne2+"}"
    txt+="\\tkzTabLine{"+ligne3+"}"
    txt+="\\end{tikzpicture}}"
    return formate(txt)

def ineq_produit(f1,f2,symbole):
    f1,f2=saisie_fonction(f1),saisie_fonction(f2)
    txt=equation_degre1(f1,"0",'>')[0].replace("\\\\","")+"\\\\"
    txt+=equation_degre1(f2,"0",'>')[0].replace("\\\\","")+"\\\\"
    txt+="On obtient ainsi le tableau suivant : \\\\"
    txt+=tableau_signe_produit(f1,f2)+"\\\\"
    f="("+latex(f1)+")*("+latex(f2)+")"
    txt+=sol_ineq(f+symbole+"0")[0]+sol_ineq(f+symbole+"0")[1]
    return txt

def ineq_quotient(f1,f2,symbole):
    f1,f2=saisie_fonction(f1),saisie_fonction(f2)
    txt=equation_degre1(f1,"0",'>')[0].replace("\\\\","")+"\\\\"
    txt+=equation_degre1(f2,"0",'>')[0].replace("\\\\","")+"\\\\"
    txt+="On obtient ainsi le tableau suivant : \\\\"
    txt+=tableau_signe_quotient(f1,f2)+"\\\\"
    f="("+latex(f1)+")/("+latex(f2)+")"
    txt+=sol_ineq(f+symbole+"0")[0]+sol_ineq(f+symbole+"0")[1]
    return txt

def sol_ineq(ineq):
    ineq=sympify(saisie_fonction(ineq))
    txt1="L'ensemble solution de $"+latex(ineq)+"$ est "
    txt="$"+latex(solve_univariate_inequality(ineq,x,relational=False))+"$"
    txt=txt.replace("(","]")
    txt=txt.replace(")","[")
    return formate(txt1),formate(txt)


	    

	
\end{pycode}